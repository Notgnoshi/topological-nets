\documentclass[12pt]{article}
% !TEX root = nets.tex

\usepackage{algorithm}
\usepackage{algpseudocode}
\usepackage{amsmath}
\usepackage{amssymb}
\usepackage{amsthm}
\usepackage[titletoc]{appendix}
\usepackage{array}
\usepackage[english]{babel}
\usepackage{booktabs}
\usepackage{cancel}
\usepackage{color}
\usepackage{eqparbox}
\usepackage{float}
\usepackage[margin=1in,left=2in]{geometry}
\usepackage{graphicx}
\usepackage[hidelinks]{hyperref}
\usepackage[utf8]{inputenc}
\usepackage{lipsum}
\usepackage{mathrsfs}
\usepackage[cache=false]{minted}
\usepackage{parskip}
\usepackage{pgfplots}
\usepackage{scalerel}
\usepackage{skull}
\usepackage{subcaption}
\usepackage{titling}
\usepackage{textcomp}
\usepackage{tikz}
\usepackage{tikz-3dplot}
\usepackage{titlesec}
\usepackage{textcomp}
\usepackage[nottoc]{tocbibind}
\usepackage[textsize=small]{todonotes}
\usepackage[normalem]{ulem}

% Document Settings

\definecolor{__minted_background_color}{rgb}{0.95, 0.95, 0.98}
\definecolor{__minted_highlight_color}{rgb}{0.88, 0.88, 1.0}
\setminted{autogobble=true,
           style=tango,
           breaklines,
           bgcolor=__minted_background_color,
           highlightcolor=__minted_highlight_color,
           mathescape, % Escape math mode everywhere.
           texcomments,  % Enable latex code inside of comments. Useful for referencing equations.
    }

\usetikzlibrary{arrows, shapes, positioning}
\pgfplotsset{compat=1.15}
\numberwithin{equation}{section}
% Sets the width of the margin TODO notes
\setlength{\marginparwidth}{1.80in}
\reversemarginpar{}

% All I want is to have comment italicized, but I cant figure out how
% to properly modify the existing \Comment macro.
% \algrenewcomment[1]{\hfill\eqparbox{COMMENT}{\textit{// #1}}}
\algnewcommand{\IComment}[1]{\Comment{\textit{#1}}}

\graphicspath{{./figures/}}

% Make \paragraph{}s end with newlines and no indents
\titleformat{\paragraph}
    {\normalfont\bfseries}
    {}
    {0pt}
    {}
% Make \paragraph{}s not have a space between the title and the text
\titlespacing*{\paragraph}{0pt}{1ex}{-\parskip}

% Document Definitions

\newcommand{\C}{\mathbb{C}}
\newcommand{\R}{\mathbb{R}}
\newcommand{\Z}{\mathbb{Z}}
\renewcommand{\O}{\mathcal{O}}

\theoremstyle{plain}
\newtheorem{thm}{Theorem}[section]

\theoremstyle{plain}
\newtheorem{lemma}[thm]{Lemma}

\theoremstyle{definition}
\newtheorem{defn}[thm]{Definition}


\theoremstyle{remark}
\newtheorem*{remark}{Remark}

\makeatletter
\def\thm@space@setup{%
%   \thm@preskip=5cm plus 1cm minus 2cm
  \thm@preskip=\parskip
  \thm@postskip=\thm@preskip % or whatever, if you don't want them to be equal
}
\makeatother

% \newtheoremstyle{definition}% name of the style to be used
%   {10pt} % measure of space to leave above the theorem. E.g.: 3pt
%   {10pt} % measure of space to leave below the theorem. E.g.: 3pt
%   {}     % name of font to use in the body of the theorem
%   {}     % measure of space to indent
%   {\bf}  % name of head font
%   {:}    % punctuation between head and body
%   { }    % space after theorem head; " " = normal interword space
%   {\thmname{#1}\thmnumber{ #2}\thmnote{ (#3)}} % Manually specify head

\renewcommand{\qedsymbol}{$\skull$}

% An inline TODO command. Doesn't play nicely with \todotableofcontents
\newcommand\todoinline[2][]{\todo[inline, caption={TODO}, #1]{
\begin{minipage}{\textwidth-4pt}#2\end{minipage}}}

% Draw clouds around things. Useful in mathematical proofs.
\newcommand{\cloud}[4][\dots]{
    \raisebox{-0.4\height}{
        \begin{tikzpicture}
            \node [cloud,
                   draw,
                   cloud puffs=#2,
                   cloud ignores aspect,
                   minimum height=#3,
                   minimum width=#4] {#1};
        \end{tikzpicture}
    }
}


\title{Nets}
\author{Austin Gill}

\begin{document}
\maketitle

We have already seen that sequences are ``adequate'' to detect limit points, continuous functions,
and compact sets in metrizable spaces. There is a generalization of the notion of a sequence,
called a \textit{net}, that will do the same for an arbitrary topological space.

\section{Directed Sets}
\begin{defn}
    A relation $\preceq$ on a set $A$ is called a \textbf{partial order} relation if the following
    conditions hold:
    \begin{enumerate}
        \item $\alpha \preceq \alpha$ for all $\alpha \in A$.
        \item If $\alpha \preceq \beta$ and $\beta \preceq \alpha$, then $\alpha = \beta$.
        \item If $\alpha \preceq \beta$ and $\beta \preceq \gamma$, then $\alpha \preceq \gamma$.
    \end{enumerate}
\end{defn}

\begin{defn}
    A \textbf{directed set} $J$ is a set with a partial order $\preceq$ such that for each pair
    $\alpha, \beta$ of elements of $J$, there exists an element $\gamma \in J$ with the property
    that
    $\alpha \preceq \gamma$ and $\beta \preceq \gamma$.
\end{defn}
Show the following are directed sets.

\begin{enumerate}
    \item Any simply ordered set under the relation $\leq$.
          \begin{proof}
              Let $A$ be a simply ordered set with order relation $<$. Show that the set $A$ is
              directed under the relation $\leq$.

              We need to show that
              \begin{enumerate}
                  \item\label{pf:part-order} $A$ is partially ordered by $\leq$.

                        Let $a, b, c \in A$.
                        \begin{enumerate}
                            \item $a \leq a$ is trivially true, because $a = a$!
                            \item Let $a \leq b$ and $b \leq a$. Because $A$ is simply ordered, $a
                                      < b$ and $b < a$ cannot hold, so the only other possibility
                                  is that $a = b$.
                            \item Let $a \leq b$ and $b \leq c$. Then clearly $a \leq c$.
                        \end{enumerate}
                        So therefore $A$ is partially ordered by $\leq$.
                  \item\label{pf:part-directed} For every $a, b \in A$ there exists a $c \in A$
                        such that $a \leq c$ and $b \leq c$.

                        Let $a, b \in A$. We wish to show that there exists a $c \in A$ with $a
                            \leq c$ and $b \leq c$.

                        $A$ is simply ordered, so either $a < b$ or $b < a$. Without loss of
                        generality, let $a < b$. But from (\ref{pf:part-order}) $A$ is partially
                        ordered by $\leq$, so we
                        have $b \leq b$!
              \end{enumerate}
              Then $\leq$ is a partial ordering on $A$ and for every pair $a, b \in A$ we have
              either $a \leq b$ and $b \leq b$, or we have $b \leq a$ and $a \leq a$! Thus $A$ is
              directed.
          \end{proof}

    \item The collection of all subsets of a set $S$, partially ordered by inclusion. That is, $A
              \preceq B$ if $A \subset B$.
          \begin{proof}
              Let $A, B \in \mathcal P(S)$. We wish to show that $\mathcal P(S)$ is directed, that
              is, that there exists a $C \in \mathcal P(S)$ with $A \subset C$ and $B \subset C$.

              Note that $S \in \mathcal P(S)$ and that $A \subset S$ and $B \subset S$. Thus
              $\mathcal P(S)$ is a directed set under inclusion.
          \end{proof}

    \item The collection $\mathcal A$ of subsets of $S$ that is closed under finite intersections,
          partially ordered by reverse inclusion. That is, $A \preceq B$ if $A \supset B$.
          \begin{proof}
              Let $\mathcal A \subset \mathcal P(S)$ such that if $A_1, \dots, A_n \in \mathcal A$
              then $\bigcap_{i=1}^n \in \mathcal A$, and let $A_1, A_2 \in \mathcal A$. We wish to
              show that $\mathcal A$ is directed. That is, that there exists a $B \in \mathcal A$
              with $A_1 \supset B$ and $A_2 \supset B$.

              But we know $\mathcal A$ is closed under finite intersections, so
              $A_1 \cap A_2 \in \mathcal A$. Moreover, we know that $A_1 \supset A_1 \cap A_2$ and
              $A_2 \supset A_1 \cap A_2$. Thus $\mathcal A$ is directed.
          \end{proof}

    \item The collection of all closed subsets of a space $X$, partially ordered by inclusion.
          \begin{proof}
              Let $\mathcal A$ be the collection of all closed subsets of a space $X$.
              Let $A, B \in \mathcal A$, that is, let $A$ and $B$ be closed in $X$.
              We wish to show that $\mathcal A$ is directed. That is, that there exists a $C \in
                  \mathcal A$ with $A \subset C$ and $B \subset C$.

              Recall that finite unions of closed sets are closed, so $A \cup B$ is closed.
              Moreover, note that $A \subset A \cup B$ and $B \subset A \cup B$.
              Thus $\mathcal A$ is directed.
          \end{proof}
\end{enumerate}

\section{Cofinal Sets}
\begin{defn}
    A subset $K \subset J$ is said to be \textbf{cofinal} in $J$ if for each $\alpha \in J$, there
    exists a $\beta \in K$ such that $\alpha \preceq \beta$.
\end{defn}

Show that if $J$ is directed, and $K$ is cofinal in $J$, then $K$ is directed.

\begin{proof}
    Let $J$ be directed and $K$ be a cofinal subset of $J$. Then for each $a, b \in J$
    there exists a $c \in J$ with $a \preceq c$ and $b \preceq c$, and for each
    $a \in J$ there exists a $\beta \in K$ such that $a \preceq \beta$. We wish
    to show that for each $\alpha, \beta \in K$ there exists a $\gamma \in K$ with
    $\alpha \preceq \gamma$ and $\beta \preceq \gamma$.

    So let $\alpha, \beta \in K$. Since $K \subset J$, $\alpha, \beta \in J$. Since $J$ is
    directed, there is a $c \in J$ with $\alpha, \beta \preceq c$. But $c \in J$ and $K$ is cofinal
    in
    $J$, so there is a $\gamma \in K$ with $c \preceq \gamma$. Since $\alpha, \beta \preceq c$ and
    $c \preceq \gamma$, we have that $\alpha, \beta \preceq \gamma$. That is, that $K$ is directed.
\end{proof}

\section{Nets and Convergence}
\begin{defn}
    Let $X$ be a topological space. A \textbf{net} in $X$ is a function $f$ from a directed set $J$
    into $X$. If $\alpha \in J$, we usually denote $f(\alpha)$ by $x_\alpha$. We denote the net $f$
    itself by ${\left(x_\alpha\right)}_{\alpha \in J}$, or simply $(x_\alpha)$ if $J$ is
    understood.
\end{defn}
\begin{defn}
    The net $(x_\alpha)$ is said to \textbf{converge} to a point $x \in X$, denoted by $x_\alpha
        \to x$ if for each neighborhood $U$ of $x$, there exists an $\alpha \in J$ such that

    \[\alpha \preceq \beta \Longrightarrow x_\beta \in U\]
\end{defn}

Show that these definitions are equivalent to those of a sequence and the convergence of a sequence
when $J = \Z_+$.

\begin{defn}
    Given a set $X$, we define a \textbf{sequence} of elements of $X$ to be a function $\vec x :
        \Z_+ \to X$.
\end{defn}

\begin{defn}
    A sequence $(x_n)$ \textbf{converges} to a point $x \in X$ if for every neighborhood $U$ of
    $x$, there is an $n^* \in \mathbb \Z_+$ such that $x_n \in U$ for all $n \geq n^*$.
\end{defn}

\begin{proof}
    Let $X$ be a topological space and $f : \Z_+ \to X$ be a net in $X$. Then clearly $f$ is a
    sequence.

    Let $X$ be a topological space and $\vec x : \Z_+ \to X$ be a sequence of elements of $X$. Then
    clearly $\vec x$ is also a net.
\end{proof}
\begin{proof}
    Let $X$ be a topological space and let $f : \Z_+ \to X$ be a net that converges to $x \in X$.
    That is, for every neighborhood $U$ of $x$, there exists an $n^* \in \Z_+$ such that
    \[ n^* \preceq n \Longrightarrow x_n \in U \]
    But the partial ordering $\preceq$ of $\Z_+$ is $\leq$, so this is precisely the definition of
    sequence convergence.
\end{proof}

\section{Convergence in a Product Space}
Suppose that $(x_\alpha)$ converges to $x$ in $X$ and $(y_\alpha)$ converges to $y$ in $Y$. Show
that $(x_\alpha \times y_\alpha)$ converges to $x \times y$ in $X \times Y$.
\begin{proof}
    \todo{Prove.}
\end{proof}

\section{Convergence in a Hausdorff Space}
Show that if $X$ is Hausdorff, a net in $X$ converges to at most one point.
\begin{proof}
    \todo{I smell contradiction.}
\end{proof}

\section{Closure of a Set}
\begin{thm}
    Let $A \in X$\todo{Is this a typo? Should it read ``Let $A \subset X$''?}{}. Then $x \in
        \overline A$ if and only if there is a net of points of $A$ converging to $x$.
\end{thm}
\begin{proof}
    \todo{To prove $\Rightarrow$, take as index set $J$ the collection of all neighborhoods of $x$,
        ordered by reverse inclusion.}
\end{proof}

\section{Continuity}
\begin{thm}
    Let $f: X \to Y$. Then $f$ is continuous if and only if for every convergent net $(x_\alpha)$
    in $X$ converging to a point $x$, the net $\big(f(a_\alpha)\big)$ converges to $f(x)$.
\end{thm}
\begin{proof}
    \todo{Prove.}
\end{proof}

\section{Subnets}
\begin{defn}
    Let $f: J \to X$ be a net in $X$; let $f(\alpha) = x_\alpha$. If $K$ is a directed set and $g:
        K \to J$ is a function such that
    \begin{enumerate}
        \item $i \preceq j \Rightarrow g(i) \preceq g(j)$
        \item $g(K)$ is cofinal in $J$
    \end{enumerate}
    then the composite function $f \circ g : K \to X$ is called a \textbf{subnet} if $(x_\alpha)$.
\end{defn}

Show that if the net $(x_\alpha)$ converges to $x$, so does any subnet of $(x_\alpha)$.
\begin{proof}
    \todo{Prove.}
\end{proof}

\section{Accumulation Points}
\begin{defn}
    Let ${(x_\alpha)}_{\alpha\in J}$ be a a net in $X$. We say that $x$ is an \textbf{accumulation
        point} of the net if for each neighborhood $U$ of $x$, the set of $\alpha$ for which
    $x_\alpha \in U$ is cofinal in $J$.
\end{defn}

\begin{lemma}
    The net $(x_\alpha)$ has the point $x$ as an accumulation point if and only if some subnet of
    $(x_\alpha)$ converges to $x$.
\end{lemma}
\begin{proof}
    \todo{To prove $\Rightarrow$, let $K$ be the set of all pairs $(\alpha, U)$ where $\alpha \in
            J$ and $U$ is a neighborhood of $x$ containing $x_\alpha$. Define $(\alpha, U) \preceq
            (\beta, V)$
        if $\alpha \preceq \beta$ and $V \subset U$. Show that $K$ is directed and use it to define
        the subnet.}
\end{proof}

\section{Compactness}
\begin{thm}
    $X$ is compact if and only if every net in $X$ has a convergent subnet.
\end{thm}
\begin{proof}
    \todo{To prove $\Rightarrow$, let $B_\alpha = \{x_\beta \mid \alpha \preceq \beta \}$ and show
        that $\{B_\alpha \}$ has the finite intersection property.

        To prove $\Leftarrow$, let $\mathcal A$ be a collection of closed sets having the finite
        intersection property, and let $\mathcal B$ be the collection of all finite intersections
        of elements of $\mathcal A$, partially ordered by reverse inclusion.}
\end{proof}

\end{document}
